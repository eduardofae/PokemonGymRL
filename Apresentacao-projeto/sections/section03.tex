\section{RL em Jogos} % Seções são adicionadas para organizar sua apresentação em blocos discretos, todas as seções e subseções são automaticamente exibidas no índice como uma visão geral da apresentação, mas NÃO são exibidas como slides separados.

%----------------------------------------------------------------------------------------

\begin{frame}
    \frametitle{Desafios}
    \begin{itemize}
        \item <1-> Conhecimento do domínio % Eu conheço as regras do jogo? Eu sei como agir dependendo da situação em que me encontro?
        \item <2-> Simplificação do problema % Talvez há uma necessidade de simplificar, pelo menos em primeira instância, o jogo que está sendo modelado.
        \item <3-> Custo computacional
    \end{itemize}
\end{frame}

%----------------------------------------------------------------------------------------
\begin{frame} 
    \frametitle{Oportunidades}
    \begin{itemize}
        \item <1-> Prova de conceito % É possível resolver o problema X utilizando o ambiente de um jogo, quais áreas serão afetadas agora, sabendo que podemos concluir o problema X com sucesso?
        \item <2-> Paralelo real x virtual % Jogos fornecem um ambiente seguro e controlado para testar algoritmos de RL antes de aplicá-los em cenários do mundo real.
        \item <3-> Simulações do mundo real % Jogos podem ser usados para criar simulações do mundo real, permitindo que os agentes de RL treinem em ambientes mais complexos e variados, tentando se preparar ao máximo na simulação para encontrar um ambiente mais complexo na realidade, como na robótica.
    \end{itemize}
\end{frame}

%----------------------------------------------------------------------------------------
