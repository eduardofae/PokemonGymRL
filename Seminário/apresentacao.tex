%%%%%%%%%%%%%%%%%%%%%%%%%%%%%%%%%%%%%%%%%
% Beamer Presentation - LaTeX Template
% Version 2.0 (March 8, 2022)
% Original Template: https://www.LaTeXTemplates.com
% Author: Vel (vel@latextemplates.com)
% License: CC BY-NC-SA 4.0

% Este modelo de apresentação foi 
% criado a partir do modelo de Giovanni Spadaro.
% Disponível em: https://github.com/Giovo17/presentation-template-unict-lm-data
%
% Adaptado por Lucas Amaral Taylor para criar uma versão especial 
% para os alunos de Matemática e Estatística da USP (IME-USP).
% Disponível em: https://github.com/lucasamtaylor01/IME-template
%%%%%%%%%%%%%%%%%%%%%%%%%%%%%%%%%%%%%%%%%

%----------------------------------------------------------------------------------------
% CLASSE DO DOCUMENTO E CONFIGURAÇÕES BÁSICAS
%----------------------------------------------------------------------------------------
\documentclass[
    11pt,               % Tamanho padrão da fonte
    % t,                % Alinhar verticalmente ao topo
    %aspectratio=169,   % Definir proporção 16:9
]{beamer}
\graphicspath{{img/}}         % Define o diretório das imagens

%----------------------------------------------------------------------------------------
% PACOTES NECESSÁRIOS
%----------------------------------------------------------------------------------------
\usepackage{
    booktabs,     % Melhora a aparência das linhas em tabelas
    palatino,     % Define Palatino como fonte principal
    subcaption    % Suporte para subfiguras
}
\usepackage[default]{opensans}  % Define Open Sans como fonte secundária
\input{config/code_langs}       % Importa configurações para highlight de código

%----------------------------------------------------------------------------------------
% CONFIGURAÇÃO DO TEMA
%----------------------------------------------------------------------------------------
% Tema Base
\usetheme{Boadilla}                            % Define o tema principal
\useinnertheme{circles}                        % Tema interno com círculos
\useoutertheme[subsection = false]{miniframes} % Tema externo com miniframes
\setbeamertemplate{navigation symbols}{}       % Remove símbolos de navegação

% Cores Personalizadas
\definecolor{primaryColor}{RGB}{20,45,105}     % Cor primária - azul escuro
\definecolor{secondaryColor}{RGB}{0,100,160}   % Cor secundária - azul médio

% Configurações de Cores
\setbeamercolor{structure}{fg=primaryColor}
\setbeamercolor{palette primary}{bg=primaryColor, fg=white}
\setbeamercolor{palette secondary}{bg=secondaryColor, fg=white}
\setbeamercolor{title}{bg=primaryColor, fg=white}

% Cores do Cabeçalho e Rodapé
\setbeamercolor{headline}{bg=secondaryColor, fg=white}
\setbeamercolor{section in head/foot}{bg=primaryColor, fg=white}
\setbeamercolor{subsection in head/foot}{bg=secondaryColor, fg=white}
\setbeamercolor{author in head/foot}{bg=primaryColor, fg=white}
\setbeamercolor{title in head/foot}{bg=secondaryColor, fg=white}
\setbeamercolor{date in head/foot}{bg=primaryColor, fg=white}
\setbeamercolor{page number in head/foot}{bg=primaryColor, fg=white}

%----------------------------------------------------------------------------------------
% BIBLIOGRAFIA
%----------------------------------------------------------------------------------------
% \usepackage{biblatex}
% \addbibresource{bibliografia.bib}

%----------------------------------------------------------------------------------------
% INFORMAÇÕES DA APRESENTAÇÃO
%----------------------------------------------------------------------------------------
\title[RL em Jogos]{Reinforcement Learning Aplicado a Jogos}          % [Versão curta]{Versão completa}
\author[Eduardo D. Faé e João P. K. Braun]{Eduardo Dalmás Faé \and João Pedro Kuhn Braun} % [Versão curta]{Nome completo}
\institute[]{Instituto de Informática - UFRGS}
\date[2025]{Novembro / 2025}

%----------------------------------------------------------------------------------------
% INÍCIO DO DOCUMENTO
%----------------------------------------------------------------------------------------
\begin{document}

% Slide de título com logo
\begin{frame}[plain,t]
    \begin{figure}
        \includegraphics[height=0.2\textheight]{UFRGS-logo.png}
        \hspace{100px}
        \includegraphics[height=0.2\textheight]{INF-logo.png}
    \end{figure}
    \titlepage
\end{frame}

% Sumário
\begin{frame}
    \frametitle{Estrutura da apresentação}
    \tableofcontents
\end{frame}

% Inclusão das seções
\section{Introdução} % Seções são adicionadas para organizar sua apresentação em blocos discretos, todas as seções e subseções são automaticamente exibidas no índice como uma visão geral da apresentação, mas NÃO são exibidas como slides separados.

%----------------------------------------------------------------------------------------
\begin{frame}
	\frametitle{Apresentação}
	\centering
  	\includegraphics[width=0.7\linewidth]{img/pokemon-3.png}
\end{frame}

%----------------------------------------------------------------------------------------
\begin{frame}
	\frametitle{Motivação}
	\begin{itemize}
		\item <1-> Padronizar o ambiente de batalha pokemon
		\item <2-> Possibilitar o treino e teste de diversos agentes nesse ambiente
	\end{itemize}
\end{frame}

%----------------------------------------------------------------------------------------
\begin{frame}
	\frametitle{Objetivo}
	\only<1>{}
	\only<2>{
		\begin{block}{}
			É possível padronizar o ambiente, o mais próximo possível do jogo oficial, para o treino e teste de agentes de aprendizado por reforço?
		\end{block}
	}
\end{frame}

%----------------------------------------------------------------------------------------

\section{Deep Learning} % Seções são adicionadas para organizar sua apresentação em blocos discretos, todas as seções e subseções são automaticamente exibidas no índice como uma visão geral da apresentação, mas NÃO são exibidas como slides separados.

%----------------------------------------------------------------------------------------
\begin{frame}
	\frametitle{DQN}
    \centering
    \includegraphics[width=0.7\linewidth]{img/dqn.png}
\end{frame}

%----------------------------------------------------------------------------------------
\begin{frame}
	\frametitle{Atari}
	\centering
    \includegraphics[width=0.6\linewidth]{img/atari.png}
\end{frame}

%----------------------------------------------------------------------------------------
\begin{frame}
	\frametitle{CNNs}
	\centering
    \includegraphics[width=0.7\linewidth]{img/cnn.png}
\end{frame}

%----------------------------------------------------------------------------------------
\begin{frame}
	\frametitle{Alpha Go}
	\centering
    \includegraphics[width=0.8\linewidth]{img/alphago.png}
\end{frame}

%----------------------------------------------------------------------------------------

\section{Domínio Contínuo} % Seções são adicionadas para organizar sua apresentação em blocos discretos, todas as seções e subseções são automaticamente exibidas no índice como uma visão geral da apresentação, mas NÃO são exibidas como slides separados.

%----------------------------------------------------------------------------------------
\begin{frame}
    \frametitle{Contextualização}
    \begin{itemize}
        \item<1-> Nem todo problema pode ser modelado como um MDP discreto;
        \item<2-> Muitos problemas no mundo real e nos jogos envolvem estados e ações contínuas.
        % \item<3-> Exemplos: 
    \end{itemize}
\end{frame}

%----------------------------------------------------------------------------------------
\begin{frame}
    \frametitle{Openai-5}
    \centering
    \includegraphics[width=0.7\linewidth]{img/openai5.png}
\end{frame}

%----------------------------------------------------------------------------------------
\begin{frame}
    \frametitle{AlphaStar}
    \centering
    \includegraphics[width=0.7\linewidth]{img/alphastar.jpg}
\end{frame}

%----------------------------------------------------------------------------------------
\begin{frame}
    \frametitle{Dreamer 4}
    \centering
    \includegraphics[width=0.8\linewidth]{img/mine-horizontal.png}
\end{frame}

%----------------------------------------------------------------------------------------

% \begin{frame} 
%     \frametitle{Comparativo} % Discreto x Contínuo
%     \begin{itemize}
%         \item <1-> Modelagem do problema % Possibilidade de modelar problema num espaço discreto? necessidade de modelar num contínuo? Prós e contras
%         \item <2-> Espaço de estados % Num espaço discreto há um número de estados finito (apesar de talvez gigantesco), já num contínuo há infinitos
%         \item <3-> Espaço de ações % Mesma coisa que para estados, algo assim
%     \end{itemize}
% \end{frame}

%----------------------------------------------------------------------------------------
\section{RL em Jogos} % Seções são adicionadas para organizar sua apresentação em blocos discretos, todas as seções e subseções são automaticamente exibidas no índice como uma visão geral da apresentação, mas NÃO são exibidas como slides separados.

%----------------------------------------------------------------------------------------

\begin{frame}
    \frametitle{Desafios}
    \begin{itemize}
        \item <1-> Conhecimento do domínio % Eu conheço as regras do jogo? Eu sei como agir dependendo da situação em que me encontro?
        \item <2-> Simplificação do problema % Talvez há uma necessidade de simplificar, pelo menos em primeira instância, o jogo que está sendo modelado.
        \item <3-> Custo computacional
    \end{itemize}
\end{frame}

%----------------------------------------------------------------------------------------
\begin{frame} 
    \frametitle{Oportunidades}
    \begin{itemize}
        \item <1-> Prova de conceito % É possível resolver o problema X utilizando o ambiente de um jogo, quais áreas serão afetadas agora, sabendo que podemos concluir o problema X com sucesso?
        \item <2-> Paralelo real x virtual % Jogos fornecem um ambiente seguro e controlado para testar algoritmos de RL antes de aplicá-los em cenários do mundo real.
        \item <3-> Simulações do mundo real % Jogos podem ser usados para criar simulações do mundo real, permitindo que os agentes de RL treinem em ambientes mais complexos e variados, tentando se preparar ao máximo na simulação para encontrar um ambiente mais complexo na realidade, como na robótica.
    \end{itemize}
\end{frame}

%----------------------------------------------------------------------------------------

\section{Pokémon} % Seções são adicionadas para organizar sua apresentação em blocos discretos, todas as seções e subseções são automaticamente exibidas no índice como uma visão geral da apresentação, mas NÃO são exibidas como slides separados.

%----------------------------------------------------------------------------------------
\begin{frame}
    \frametitle{Contextualização}
    \only<1>{
        \centering
        \includegraphics[width=0.8\linewidth]{img/pokemon-3.png}
    }
    \only<2>{
        \centering
        \includegraphics[width=0.7\linewidth]{img/pokemonbattle.png}
    }
    \only<3>{
        \centering
        \includegraphics[width=0.5\linewidth]{img/pokemon-type-chart.png}
    }
    \only<4>{
        \centering
        \includegraphics[width=0.7\linewidth]{img/volcarona.png}
    }
\end{frame}

%----------------------------------------------------------------------------------------
\begin{frame}
    \frametitle{Características}
    \begin{itemize}
        \item <1-> Atualizações constantes % Diferente de um jogo clássico com regras já bem definidas que não serão alteradas, pokemon é um jogo vivo que recebe novas atualizações com novas mecânicas, pokémons, itens, etc. 
        \item <2-> Combate em turnos simultâneos % O combate em turno do pokemon traz uma característica diferente que damas, xadrez ou go, etc, não possuem, a qual o turno é jogado simultaneamente, então o resultado observado de uma ação depende não somente da ação do jogador, como também de seu oponente.
        \item <3-> Diversidade de estratégias % Existem mais de 1000 pokemons, e com 6 espaços na equipe para batalhar, quais desses pokemons foram escolhidos? Destes, quais movimentos cada um possui, que não necessariamente está preso em um padrão?
    \end{itemize}
\end{frame}

%----------------------------------------------------------------------------------------
% \begin{frame}
%     \frametitle{Comparativo}
%     % TO DO
% \end{frame}

%----------------------------------------------------------------------------------------

\section{Conclusão} % Seções são adicionadas para organizar sua apresentação em blocos discretos, todas as seções e subseções são automaticamente exibidas no índice como uma visão geral da apresentação, mas NÃO são exibidas como slides separados.

%----------------------------------------------------------------------------------------
\begin{frame}
    \frametitle{Conclusão}
    \begin{itemize}
        \item <1-> Duas novas \textit{features}
        \item <2-> PPO e DQN aprenderam a lidar com a vantagem
        \item <3-> RPPO se saiu melhor no geral em ambientes \textit{meta-aware}
    \end{itemize}
\end{frame}

%----------------------------------------------------------------------------------------
\begin{frame}
    \frametitle{Limitações}
    \begin{itemize}
        \item <1-> Tempo de pesquisa
        \item <2-> \textit{Features} cortadas
        \item <3-> Códigos deprecados
        \item <4-> Tempo de treinamento
    \end{itemize}
\end{frame}

%----------------------------------------------------------------------------------------
\begin{frame}
    \frametitle{Trabalhos Futuros}
    \begin{itemize}
        \item <1-> Adicionar habilidades que garantem imunidade
        \item <2-> Adicionar velocidade
        \item <3-> Implementar agentes próprios
        \item <4-> Expandir para mais funcionalidades
    \end{itemize}
\end{frame}

%----------------------------------------------------------------------------------------
\section{Pokémon Gym} % Seções são adicionadas para organizar sua apresentação em blocos discretos, todas as seções e subseções são automaticamente exibidas no índice como uma visão geral da apresentação, mas NÃO são exibidas como slides separados.

%----------------------------------------------------------------------------------------
\begin{frame}
    \frametitle{Apresentação}
\end{frame}

%----------------------------------------------------------------------------------------
\begin{frame}
    \frametitle{Simplificações}
\end{frame}

%----------------------------------------------------------------------------------------
\begin{frame}
    \frametitle{Resultados}
\end{frame}

%----------------------------------------------------------------------------------------
\begin{frame}
    \frametitle{Oportunidades}
\end{frame}

%----------------------------------------------------------------------------------------

% Referências
% \begin{frame}{Referências}
%     \nocite{*}
%     \printbibliography[heading=none]
% \end{frame}

% Slide final
\section*{}
\begin{frame}[plain, noframenumbering]
    \begin{center}
        {\Huge Obrigado pela Atenção!}
    \end{center}
\end{frame}

\end{document}


