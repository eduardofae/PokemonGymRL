\section{Domínio Contínuo} % Seções são adicionadas para organizar sua apresentação em blocos discretos, todas as seções e subseções são automaticamente exibidas no índice como uma visão geral da apresentação, mas NÃO são exibidas como slides separados.

%----------------------------------------------------------------------------------------
\begin{frame}
    \frametitle{Contextualização}
    \begin{itemize}
        \item<1-> Nem todo problema pode ser modelado como um MDP discreto;
        \item<2-> Muitos problemas no mundo real e nos jogos envolvem estados e ações contínuas.
        % \item<3-> Exemplos: 
    \end{itemize}
\end{frame}

%----------------------------------------------------------------------------------------
\begin{frame}
    \frametitle{Openai-5}
    \centering
    \includegraphics[width=0.7\linewidth]{img/openai5.png}
\end{frame}

%----------------------------------------------------------------------------------------
\begin{frame}
    \frametitle{AlphaStar}
    \centering
    \includegraphics[width=0.7\linewidth]{img/alphastar.jpg}
\end{frame}

%----------------------------------------------------------------------------------------
\begin{frame}
    \frametitle{Dreamer 4}
    \centering
    \includegraphics[width=0.8\linewidth]{img/mine-horizontal.png}
\end{frame}

%----------------------------------------------------------------------------------------

% \begin{frame} 
%     \frametitle{Comparativo} % Discreto x Contínuo
%     \begin{itemize}
%         \item <1-> Modelagem do problema % Possibilidade de modelar problema num espaço discreto? necessidade de modelar num contínuo? Prós e contras
%         \item <2-> Espaço de estados % Num espaço discreto há um número de estados finito (apesar de talvez gigantesco), já num contínuo há infinitos
%         \item <3-> Espaço de ações % Mesma coisa que para estados, algo assim
%     \end{itemize}
% \end{frame}

%----------------------------------------------------------------------------------------