\documentclass{article}


\usepackage{PRIMEarxiv}
\usepackage[utf8]{inputenc} % allow utf-8 input
\usepackage[T1]{fontenc}    % use 8-bit T1 fonts
\usepackage{hyperref}       % hyperlinks
\usepackage{url}            % simple URL typesetting
\usepackage{booktabs}       % professional-quality tables
\usepackage{amsfonts}       % blackboard math symbols
\usepackage{nicefrac}       % compact symbols for 1/2, etc.
\usepackage{microtype}      % microtypography
\usepackage{lipsum}
\usepackage{graphicx}
\usepackage[round,authoryear]{natbib}
\graphicspath{{media/}}     % organize your images and other figures under media/ folder

  
%% Title
\title{Simulador de Batalhas Pokemon}

\author{
  Eduardo Dalmás Faé, João Pedro Kuhn Braun \\
  Instituto de Informática \\
  Universidade Federal do Rio Grande do Sul \\
  Porto Alegre\\
  \texttt{\{edfae, jpkbraun\}@inf.ufrgs.br} \\
}


\begin{document}
\maketitle


\begin{abstract}
\lipsum[1]
\end{abstract}

% keywords can be removed
\keywords{First keyword \and Second keyword \and More}


\section{Introduction}

a



\section{Conceitos Básicos}

Pokemon é uma franquia da Nintendo que começou como um jogo no ano de 1996, que com o passar dos anos evoluiu para muitas outras categorias de conteúdo, tornando-se mundialmente reconhecida. Entre muitas das dinâmicas acerca do jogo, que trata desde narrativa de uma história até explorações de cenários e segredos, pode-se afirmar que uma das mecânicas \textit{core} que traz vida ao jogo são as batalhas Pokemon.

Pokemons são seres vivos nesse mundo de fantasia que se assemelham com animais, e treinadores desses pokemons os capturam e usam para batalhar entre si, dinâmica essa importantíssima e muito levada a sério nesse mundo. Esses seres vivos, no contexto da batalha, possuem tipos, quatro movimentos, cada um possuindo tipos e poderes, atributos que ditam a sua "qualidade", etc. A batalha ocorre em turnos em que, simultaneamente, cada treinador pokemon decide se deseja trocar o pokemon atual para outro de sua equipe, se possível, ou atacar com um de seus 4 movimentos válidos. Caso um único treinador decida trocar, a troca ocorre primeiro, caso ambos troquem, a ordem não importa, e se ambos pokemons atacarem, executa sua ação primeiro quem é o mais veloz.




\section{Trabalhos Relacionados}

\textbf{Batalha Pokemon} Trabalhos anteriores já modelaram ambientes de batalha pokemon para treino e teste de agentes. \citet{poke-battle} definiram um ambiente simplificado de pokemon para iniciar seus experimentos, reduzindo enormemente a complexidade do jogo para tal. Para o treino de seus agentes, foram utilizados dois algoritmos, WPL (\textit{Weighted Policy Learner}) \cite{wpl} e GIGAWoLF (\textit{Generalized Infinitesimal Gradient Ascent Win or Learn Fast}) \cite{gigawolf} em cima de uma configuração de times pokemons semi-aleatória, para então serem testados em um ambiente determinístico criado pelos próprios autores, para descobrir se os agentes aprenderam a executar uma ação ruim a curto prazo, trocar de pokemon, para uma possível vitória num horizonte mais distante. O presente trabalho utilizou de um ambiente mais complexo, além de definir mais agentes, com diferentes algoritmos para seus treinamentos.

\textbf{Padronização de Batalhas} Houveram tentativas de padronizar o cenário geral de batalha e modelação de um ambiente pokemon, além de permitir a integração de agentes treinados com simuladores de batalhas pokemon grandemente utilizados hoje em dia, como \href{https://pokemonshowdown.com/}{Pokemon Showdown}. Entretanto, trabalhos como os de \cite{poke-env} falham hoje em executar esse proposta com êxito, visto alguns problemas de uso e falta de documentação adequada. Esta pesquisa não integra o agente treinado com simuladores para interação e teste com outros jogadores, porém traz uma padronização não só de um ambiente de batalha pokemon através do uso do \href{https://pettingzoo.farama.org/}{PettingZoo} \cite{pettingzoo}, mas também traz padronização no treino e uso de agentes.


\section{Metodologia}

a



\section{Resultados e Discussão}

a



\section{Conclusão}

a



%Bibliography
\bibliographystyle{plainnat}
\bibliography{references}  


\end{document}
