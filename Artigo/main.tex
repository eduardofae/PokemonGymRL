\documentclass{article}


\usepackage{PRIMEarxiv}

\usepackage[utf8]{inputenc} % allow utf-8 input
\usepackage[T1]{fontenc}    % use 8-bit T1 fonts
\usepackage{hyperref}       % hyperlinks
\usepackage{url}            % simple URL typesetting
\usepackage{booktabs}       % professional-quality tables
\usepackage{amsfonts}       % blackboard math symbols
\usepackage{nicefrac}       % compact symbols for 1/2, etc.
\usepackage{microtype}      % microtypography
\usepackage{lipsum}
\usepackage{graphicx}
\graphicspath{{media/}}     % organize your images and other figures under media/ folder

  
%% Title
\title{Simulador de Batalhas Pokemon
%%%% Cite as
%%%% Update your official citation here when published 
\thanks{\textit{\underline{Citation}}: 
\textbf{Authors. Title. Pages.... DOI:000000/11111.}} 
}

\author{
  Eduardo Dalmás Faé, João Pedro Kuhn Braun \\
  Instituto de Informática \\
  Universidade Federal do Rio Grande do Sul \\
  Porto Alegre\\
  \texttt{\{edfae, jpkbraun\}@inf.ufrgs.br} \\
}


\begin{document}
\maketitle


\begin{abstract}
\lipsum[1]
\end{abstract}

% keywords can be removed
\keywords{First keyword \and Second keyword \and More}


\section{Introduction}

a



\section{Conceitos Básicos}

Pokemon é uma franquia da Nintendo que começou como um jogo no ano de



\section{Trabalhos Relacionados}

a



\section{Metodologia}
Your conclusion here



\section{Resultados e Discussão}
This was was supported in part by......



\section{Conclusão}

a



%Bibliography
\bibliographystyle{unsrt}  
\bibliography{references}  


\end{document}
